\documentclass[a4paper, 10pt]{article}
\usepackage{amsmath}
\usepackage{amsthm}
\usepackage{amssymb}	
\usepackage{graphicx}
\usepackage{mathtools}
\usepackage{multirow}
\usepackage{hyperref}
\usepackage{natbib}

\topmargin 0.0cm
\oddsidemargin 0.2cm
\textwidth 16cm  
\textheight 21cm
\footskip 1.0cm
\linespread{1.5}  


% Citation styles
\bibliographystyle{apalike}
\setcitestyle{authoryear, open={(}, close={)}}

\begin{document}
%-------------------------------------------------------------------------------------------------
%--- Title
%-------------------------------------------------------------------------------------------------
\begin{center}
    \Huge
    Proto-word reconstruction with NNs
\end{center}
%-------------------------------------------------------------------------------------------------
%--- Prev work
%-------------------------------------------------------------------------------------------------
\section{Work to build on}
\begin{itemize}
    \item Reversible-jump MCMC methods \cite{bouchard-Cote_et_al:2013} 
    \item Conditional random fields \& RNN \cite{ciobanu_ab_2018}
    \item RNNs \cite{meloni_ab_2019} 
\end{itemize}

%-------------------------------------------------------------------------------------------------
%--- Data
%-------------------------------------------------------------------------------------------------
\section{Data sources} 
\begin{itemize}
    \item Wiktionary 
    \begin{itemize}
        \item Many datapoints of dubious quality
        \item Would have to do most of the data extraction ourselves
        \item Data dumps available \href{https://dumps.wikimedia.org/enwiktionary/}{here}
        \item Technically it should be possible possible to get \href{https://wiki.dbpedia.org/wiktionary-rdf-extraction}{RDF data}
    \end{itemize}
    \item \href{http://ielex.mpi.nl/}{Indo-European lexical cognacy database}
    \begin{itemize}
        \item Used by famous \cite{bouckaert_et_al:2012}
        \item No longer maintained (since 2016)
        \item Only Indo-European data, which may be a bit over-investigated
        \item But: plain TSV
    \end{itemize}
    \item \href{https://starling.rinet.ru/downl.php?lan=en}{Evolution of human language project} (used in \cite{hruschka_detecting_2015}).
    provides cognate data for several Eurasian language families (Altaic, Tungusic, Mongolic, Japonic...)
    \begin{itemize}
        \item I didn't know the format (dBase/.dbf), don't know exactly how to use 
        \item Somewhat outdated (2013)
        \item Pro: Many languages from many families
        \item Could try to reconstruct proto-Altaic (which is a deprecated clade) or proto-Transeurasian
    \end{itemize}
\end{itemize}

%-------------------------------------------------------------------------------------------------
%--- Model
%-------------------------------------------------------------------------------------------------
\section{Model architectures}
\begin{itemize}
    \item Code letters for phonological features
    \begin{itemize}
        \item Word = $n_{letters} \times n_{features}$ array
        \newline
        \newline
        $\begin{Bmatrix}
        l/f & \pm cons & \pm vow & \pm cont & \pm front & .. \\
        t & 1 & 0 & 0 & 1
        \end{Bmatrix}$
    \end{itemize}
\end{itemize}

%-------------------------------------------------------------------------------------------------
%--- Refs
%-------------------------------------------------------------------------------------------------
\bibliography{../../bib/NLPwithNN.bib}
\end{document}